\section{Tool requirements}

\subsection{Overview}
The following list of requirements are formed from the defined set of tool requirements in the coursework brief and based on our algorithm choice, JPlag plagiarism detection.
We have chosen to implement our tool using the Greedy String Tiling algorithm as described in the JPlag paper, because from the research and comparison of the different methods, the JPlag process and algorithm proved to be the most coherent and logical in terms of the steps that were involved. Furthermore, JPlag has proved to be an efficient algorithm, being able to process hundreds of lines of source code when given more than a hundred programs as input. Thus, as we are required to only give two files as input we were certain that performance would not be an issue using this approach. In addition, JPlag ignores syntactic sugar such as comments and white space as well as names of identifiers to make it more accurate in detecting clones within the input programs.
\subsection{Requirements}
\begin{enumerate}
\item The tool will compute a similarity measurement for two input Java source code files using an implementation of the JPlag plagiarism detection algorithm.

\item The tool must work via command line and should not have a graphical interface.

\item Input taken should be the names of two files as command lines arguments.

\item The output should be a similarity measure to STDOUT as a percentage.

\item Input Java source files will be parsed using a Java parser library, ANTLR.

\item The tool will be written in Java (version 6) to meet the requirements for the parser library.

\item The tool will not invoke the JPlag binary or any other similarity testing tool.

\item The output of similarity measurement by plagiarism detection algorithm will complete in reasonable time for expected inputs. Expected inputs are generally Java source code files and specifically under testing TOH.java, variants of TOH.java and arbitrary dissimilar test inputs.

\item The tool will have an automated testing mechanism to carry out pairwise comparison of the said files.

\item The tool should work with all platforms the JVM works on that support ANTLR. The tool will be tested and run on Linux and Windows platforms.

\end{enumerate}
\break
